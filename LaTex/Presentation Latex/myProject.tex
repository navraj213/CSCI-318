\documentclass[pdf]{beamer}
\usetheme{Warsaw}
\usecolortheme{default}
\usepackage{graphicx}
\usepackage{times}
\usepackage{chancery}
\mode<presentation>{}
%% preamble
\title{Formatting in Beamer}
\author[]{By Benoy Thomas, Navraj Singh, Bikram Singh}
\begin{document}
\maketitle
\begin{frame}{Table of Contents}
    \tableofcontents
\end{frame}
\section{Formatting in Latex}
\begin{frame}{Formatting in LaTeX}
    The thing that makes Latex so useful is the fact that
    there are so many in built packages that you can utililize.

    \par Such as:
    \begin{itemize}
        \item {\fontfamily{pzc}\selectfont Font Families}
        \item \Huge{Font} \huge{Styles} \Large{or} \large{Sizes}
        \item \textbf{Bold,} \textit{italics,} \underline{ and Underlining}
        \item Mathematical $\Sigma$xpressions
    \end{itemize}

\end{frame}

\begin{frame}{Font Families}
    \par By default, in standard LaTeX classes the default style for text is usually a Roman (upright) serif font. 
    \par To change the font you just need to use some simple latex commands
    \par Latex has hundreds thousands of different fonts you can choose from.
    \par The code would look like:
    \begin{columns}
        \begin{column}{0.5\textwidth}
            \begin{center}
                CODE
            \end{center}
            \par \small{\textbf{\textit{\texttt{\textbackslash usepackage\{chancery\}}}}}
            \par \small{\textbf{\texttt{\textbackslash fontfamily\{pzc\}\textbackslash selectfont Hello in Cursive}}}
        \end{column}
        \vrule
        \begin{column}{0.5\textwidth}
            \begin{center}
                OUTPUT
            \end{center}
            \par \small{\fontfamily{pzc}\selectfont \,Hello in Cursive\\~\\~\\}
        \end{column}
    \end{columns}
\end{frame}

\begin{frame}{Font Styles and Sizes}
    LaTeX offers various options for customizing the 
    appearance of your text. While it automatically 
    sets fonts based on document structure, you can 
    control both size and style. Default font size is 
    10pt, with options ranging from 8pt to 20pt. 
    Commands like tiny and large allow for specific 
    size adjustments.
    \begin{columns}
        \begin{column}{0.5\textwidth}
            \begin{center}
                CODE\\~\\
            \end{center}
            \par \small{\textbf{\textit{\texttt{\textbackslash Huge\{Huge Text\}}}}}
            \par \small{\textbf{\textit{\texttt{\textbackslash tiny\{Tiny Text\}}}}}
        \end{column}
        \vrule
        \begin{column}{0.5\textwidth}
            \begin{center}
                OUTPUT
            \end{center}
            \par \, \Huge{Huge Text}
            \par \, \tiny{Tiny Text}
        \end{column}
    \end{columns}

\end{frame}
\begin{frame}{Bold, Italics, and Underlining}
    LaTeX provides distinct commands for achieving bold, italic, 
    and underlined text, allowing you to emphasize 
    specific words or phrases. To make text bold,
     use the textbf command. For italics, 
     use textit or emph. Underlining can be 
     achieved with underline. These commands 
     can be nested to combine effects. For example, \textbf{\textit{important}} 
     produces bold and italicized text. However, overuse
     of these formatting options can hinder readability.

     \begin{columns}
        \begin{column}{0.5\textwidth}
            \begin{center}
                CODE
            \end{center}
            \par \footnotesize{\textbf{\textit{\texttt{\textbackslash textbf\{Bold,\}}}}}
            \par \footnotesize{\textbf{\textit{\texttt{\textbackslash textit\{Italics,\}}}}}
            \par \footnotesize{\textbf{\textit{\texttt{\textbackslash underline\{and Underline.\}}}}}
            \par \footnotesize{\textbf{\textit{\texttt{\textbackslash textbf\{\textbackslash textit\{\textbackslash \\ underline\{All.\}\}\}}}}}
        \end{column}
        \vrule
        \begin{column}{0.5\textwidth}
            \begin{center}
                OUTPUT
            \end{center}
            \par \, \textbf{Bold,} 
            \par \, \textit{Italics,} 
            \par \, \underline{and Underline.}
            \par \, \textbf{\textit{\underline{All.}}}
        \end{column}
    \end{columns}
    
\end{frame}
\begin{frame}{Mathmatical Expressions}
        What's even crazier is that Latex can be used for Mathmatical expressions as well.
        There are two modes for Mathmatical equations.
        \begin{itemize}
            \item Inline Mode: is when you need to make a short expression within regular text
            \item Display mode: is for when you want to represent stand alone equations (you can do this [ ] or with the begin {equation} )
        \end{itemize}
\end{frame}

\begin{frame}{Mathmatical Examples}
    \par \textbf{\textit{Inline Mode}}
    \par \textbf{\textit{CODE:}} {\textit{\texttt{\$a\^{}2 + b\^{}2 = c\^{}2\$}}}
    \par \textbf{\textit{OUTPUT:}} $a^2 + b^2 = c^2$
    \par \textbf{\textit{\\Display Mode}}
    \par \textbf{\textit{CODE:}} {\textit{\texttt{\textbackslash begin\{equation\} \\ a\^{}2 + b\^{}2 = c\^{}2 \\ \textbackslash end\{equation\}}}}
    \par \textbf{\textit{\\OUTPUT:}} \begin{equation} a^2 + b^2 = c^2 \end{equation}
\end{frame}



\section{Formatting in Beamer}
\begin{frame}{What is Beamer?}
    Beamer is a LaTeX document class for 
    creating presentation slides, with a 
    wide range of templates and a set of 
    features for making slideshow effects. 
\end{frame}

\begin{frame}{Why is it useful?}
    Beamer is useful becuase of the wide range of packages that are provided in it.
    Compared to languages like html you don't have to worry about styling due to the fact that
    a lot of the styles come imbedded in the beamer.
\end{frame}
\begin{frame}{Color Themes}
    Beamer provides a diverse range of color themes when it comes to slides, some of which include:
    \begin{enumerate}
        \item seams: Creates a modern and professional look with subtle color gradients and contrasting elements.
        \item Warsaw: Utilizes bold colors and clear divisions between presentation elements, making it suitable for impactful presentations.
        \item adrid: Provides a clean and minimalist aesthetic with muted tones, ideal for presentations emphasizing text and content.
        \item ontpellier: Offers a vibrant and colorful scheme, well-suited for presentations targeting younger audiences or aiming for a more informal tone.
        \item PaloAlto: Features a dark background with bright accent colors, perfect for highlighting key points and creating a dramatic effect.
    \end{enumerate}
        
\end{frame}
\begin{frame}{Frame Types}
    \par \textbf{\textit{Frame Types:}} 
\par Frame: The basic building block, containing the main content of your presentation.
\par Title: Used for the presentation title and author information.
\par Slide: Represents individual slides within a frame, allowing for multi-step presentations.
\par Navigation: Displays navigation elements like previous/next buttons and frame thumbnails.
\par \textbf{\textit{Navigation styles:}} 
\par Navigation symbols: The default style, showing small circles or squares representing frames.
\par Scrollbar: Displays a horizontal scrollbar for navigating through frames.
\par Miniature: Presents thumbnail previews of each frame for quick visual navigation.
\par Outline: Shows a hierarchical outline of the presentation structure.
\end{frame}
\begin{frame}{Customization}
    Beamer allows for a lot of customizing. Where you can customize your own themes in \textbf{\textit{.sty files}} 
    which can then be loaded into within your slides.
    \par Some advanced features allow for overlays and animations to be added to certain slides.
\end{frame}
\begin{frame}{}
    \centering \Large
    \emph{Questions?}
  \end{frame}


\end{document}